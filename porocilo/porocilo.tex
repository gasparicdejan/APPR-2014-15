\documentclass[11pt,a4paper]{article}

\usepackage[slovene]{babel}
\usepackage[utf8x]{inputenc}
\usepackage{graphicx}

\pagestyle{plain}

\begin{document}
\title{Poročilo pri predmetu \\
Analiza podatkov s programom R}
\author{Študent FMF}
\maketitle

\section{Izbira teme}
Vsebina, ki jo bom obravnaval pri tem projektu obsega trgovanje Slovenije. V obsegu te analize, bom obravnaval izvoz ter uvoz naše države z drugimi svetovnimi državami v zadnjih letih. Za vsako državo s katero Slovenija trguje, bom prikazal podatke po vseh izdelkih.

Podatke bom pridobil na spletni strani: http://wits.worldbank.org/CountryProfile/Country/SVN/Year/2013/Summary, iz katere bom podatke prenesel v program Exel ter jih nato uvozil v program R.

Namen analiziranja podatkov je ugotoviti, kako se spreminjajo količina uvoženih in izvoženih količin, kakšna je razlika med uvozom in izvozom z vsako posamezno državo, ter kako se spreminja skozi čas.

\section{Obdelava, uvoz in čiščenje podatkov}

V tej fazi sem uvozil šest tabel. S spletne strani http://wits.worldbank.org/CountryProfile/Country/SVN/Year/2013/Summary sem najprej podatke uvozil v Exel, nato pa od tam v R.

Podatki zavzemajo količine uvoženih te izvoženih količin od leta 1999 do leta 2013 za posamezne države.

\section{Analiza in vizualizacija podatkov}

\includegraphics{../slike/povprecna_druzina.pdf}

\section{Napredna analiza podatkov}

\includegraphics{../slike/naselja.pdf}

\end{document}
